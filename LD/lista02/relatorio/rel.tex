\documentclass[12pt, ar4paper]{article}
\usepackage[hmargin=2.5cm,vmargin=3cm,bmargin=3cm]{geometry}
\usepackage{indentfirst}
\usepackage{graphicx}
\usepackage[brazil]{babel}  % pt_BR hifenization
\usepackage[utf8]{inputenc}
\usepackage{titling}
\newcommand{\subtitle}[1]{%
  \posttitle{%
    \par\end{center}
    \begin{center}\large#1\end{center}
    \vskip0.3em}%
}

\begin{document}

\title{\textbf{Linguagens Declarativas}}
\subtitle{\textbf{Segunda lista de exercícios}}
\author
{
	Daniella Angelos\\
	11/0010434
}
\date{}
\maketitle

\section*{Sobre este relatório}

Documento destinado à apresentação de um resumo das soluções implementadas para os dois exercícios da segunda lista da disciplina.

\section*{Dos exercícios propostos}

\subsection*{1. Validação de cartões de crédito}

Um dos mais utilizados algoritmos para checagem de validação de números de cartões de crédito é conhecido por \textit{Luhn Algorithm}.

O arquivo que corresponde a este exercício é o \textit{ValidacaoCartao.hs}. 

\subsection*{2. Mini linguagem de programaçao funcional}

\section*{Conclusão}

A implementa\c{c}\~ao desse display gr\'afico mapeado em mem\'oria permitiu melhor familiariza\c{c}\~ao com programa\c{c}\~ao Assembler, al\'em de melhor entendimento do endere\c{c}amento da mem\'oria, retorno de fun\c{c}\~oes e, como um conhecimento adicional, melhor entendimento de algoritmos de desenho de figuras geom\'etricas simples, que, usando recursos e criatividade, podem se tornar algo mais complexo, ou ser \'util para outros tipos de implementa\c{c}\~ao. 

\end{document}